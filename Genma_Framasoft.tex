\documentclass{beamer}
\mode<presentation> {
%\usetheme{Madrid}
%\usetheme{default}
\usepackage{color}
\definecolor{bottomcolour}{rgb}{0.21,0.11,0.21}
\definecolor{middlecolour}{rgb}{0.21,0.11,0.21}
\setbeamercolor{structure}{fg=white}
\setbeamertemplate{frametitle}[default]%[center]
\setbeamercolor{normal text}{bg=black, fg=white}
\setbeamertemplate{background canvas}[vertical shading]
[bottom=bottomcolour, middle=middlecolour, top=black]
\setbeamertemplate{items}[circle]
\setbeamertemplate{navigation symbols}{} %no nav symbols
\setbeamercolor{block title}{use=structure,fg=white,bg=structure.fg!50!red!50!blue!100!green}
\setbeamercolor{block body}{parent=normal text,use=block title,bg=block title.bg!5!white!10!bg,fg=white}
\setbeamertemplate{navigation symbols}{}
}

\usepackage{graphicx} 
\usepackage{booktabs} 
\usepackage[utf8]{inputenc}  
\usepackage[T1]{fontenc}  
\usepackage{geometry}     
\usepackage[francais]{babel} 
\usepackage{eurosym}
\usepackage{verbatim}
\usepackage{ragged2e}
\justifying

\input{cc_beamer}

\title[Framasoft et la dégooglisation]{Framasoft et la dégooglisation} 
\author{Genma}

\begin{document}

%% Titlepage
\begin{frame}
	\titlepage
	\vfill
	\begin{center}
		\CcGroupByNcSa{0.83}{0.95ex}\\[2.5ex]
		{\tiny\CcNote{\CcLongnameByNcSa}}
		\vspace*{-2.5ex}
	\end{center}
\end{frame}


%----------------------------------------------------------------------------------------

\begin{frame}
\frametitle{\includegraphics[scale=0.4]{./images/Genma.jpg} \ \ \  A propos de moi  }
\begin{columns}[c] 

\column{.55\textwidth} 
\textbf{Où me trouver sur Internet?}
\begin{itemize}
\item Le Blog de Genma : http://genma.free.fr
\item Twitter : http://twitter.com/genma
\end{itemize}

\textbf{Ce que je fais?}
\\ Plein de choses dont:
\begin{itemize}
\item Des "chiffrofêtes"
\item Framasoft
\end{itemize}

\column{.5\textwidth} 
\includegraphics[width=5cm,height=5cm]{./images/blog.png} 
\end{columns}
\end{frame}

%----------------------------------------------------------------------------------------
\begin{frame}
\Huge{\centerline{Framasoft}}
\end{frame}


%----------------------------------------------------------------------------------------
\begin{frame}
\Huge{\centerline{Framasoft et la dégooglisation}}
\end{frame}

\begin{frame}
\begin{center}
\includegraphics[scale=0.5]{./images/DegooglisonsInternet.jpg}
\end{center}
\end{frame}
%------------------------------------------------
\begin{frame}
\frametitle{Dégooglisons Internet}

\begin{block}{Framasoft lance une campagne d’envergure : Dégooglisons Internet. }
\justifying{
Après avoir mis en place Framapad, Framadate, Framindmap, Framanews et le petit dernier Framabag, il était temps de passer à la vitesse supérieure en annonçant la mise en place de plus de services L.E.D.S. (Libres, Éthiques, Décentralisés et Solidaires). 
\\~\\L’annonce s’est donc accompagnée de l’ouverture d’un nouveau pod Diaspora* pour les francophones soucieux de leur vie privée : la Framasphère.
}
\end{block}
\end{frame}

%------------------------------------------------
\begin{frame}
\frametitle{Dégooglisons Internet}

\begin{block}{L'objectif}
\justifying{
Installer, face à chaque service propriétaire, un service L.E.D.S. hébergé par les soins de Framasoft. 
\\~\\L’association s’inscrit dans un contexte d’ouverture en encourageant les projets qui participeront à cet effort d’émancipation des « grands de l’Internet ».
}
\end{block}
Plius de détail sur \url{http://degooglisons-internet.org/}
\end{frame}


%------------------------------------------------
\begin{frame}
\frametitle{Enjeux}

\begin{block}{Les enjeux}
\justifying{
\begin{itemize}
\item Concentration des acteurs d’Internet autour de sillos
\item Une centralisation nuisible (frein à l'innovation)
\item  Les utilisateurs de ces services derniers ne contrôlent plus leur vie numérique
\end{itemize}
}
\end{block}
Plius de détail sur \url{http://degooglisons-internet.org/}
\end{frame}

%------------------------------------------------
\begin{frame}
\frametitle{Dangers}

\begin{block}{Les dangers}
\justifying{
Les services en ligne toujours plus centralisés de géants tentaculaires comme Google, Amazon, Facebook, Apple ou Microsoft (GAFAM) mettent en danger nos vies numériques.
\begin{itemize}
\item Espionnage
\item Vie privée
\item Centralisation
 \item Fermeture
\end{itemize}
}
\end{block}

Plius de détail sur \url{http://degooglisons-internet.org/}
\end{frame}

%------------------------------------------------
\begin{frame}
\frametitle{Les propositions de Framasoft}

\begin{block}{Ce que Framasoft propose}
\justifying{
Framasoft souhaite faire face à ces dangers menaçant nos vies numériques en proposant des services
\begin{itemize}
\item  libres,
\item éthiques, 
\item décentralisés 
\item et solidaires.
\end{itemize}
}
\end{block}
Plius de détail sur \url{http://degooglisons-internet.org/}
\end{frame}

%------------------------------------------------
\begin{frame}
\frametitle{Concrètement, ce que fait Framasoft}

\begin{block}{Concrètement}
Le projet « Dégooglisons Internet » - qui ne concerne d'ailleurs pas que Google - consiste à proposer des services alternatifs face à un maximum de services que nous évaluons comme menaçants pour nos vies numériques.
\justifying{
\begin{itemize}
\item Des services sont libres, gratuits, ouverts à tous (dans la limite de nos capacités techniques et financières),
\item Promotion de l'auto-hébergement, 
\item Proposer une alternative.
\end{itemize}
}
\end{block}
Plius de détail sur \url{http://degooglisons-internet.org/}
\end{frame}

%----------------------------------------------------------------------------------------
\begin{frame}
\Huge{\centerline{Les services "cloud" de Framasoft}}
\end{frame}
%------------------------------------------------

\begin{frame}
\begin{center}
\includegraphics[scale=0.4]{./images/Framapad.jpg}
\end{center}
\end{frame}

\begin{frame}
\begin{center}
\includegraphics[scale=0.4]{./images/FramaCalc.jpg}
\end{center}
\end{frame}

\begin{frame}
\begin{center}
\includegraphics[scale=0.4]{./images/Framadate.jpg}
\end{center}
\end{frame}

\begin{frame}
\begin{center}
\includegraphics[scale=0.4]{./images/Framavectoriel.jpg}
\end{center}
\end{frame}

\begin{frame}
\begin{center}
\includegraphics[scale=0.4]{./images/Framindmap.jpg}
\end{center}
\end{frame}

\begin{frame}
\Huge{\centerline{Plein d'autres}}
\end{frame}

\begin{frame}
\begin{center}
\includegraphics[scale=0.5]{./images/Roadmap.jpg}
\end{center}
\end{frame}


%----------------------------------------------------------------------------------------
\begin{frame}
\Huge{\centerline{Soutenir Framasoft}}
\end{frame}
%------------------------------------------------

\begin{frame}
\begin{center}
\includegraphics[scale=0.5]{./images/Soutien01.jpg}
\end{center}
\end{frame}

%------------------------------------------------
\begin{frame}
\frametitle{Pourquoi soutenir Framasoft}

\begin{block}{6 bonnes raisons de nous soutenir}
\justifying{
\begin{itemize}
\item Parce que l’enfermement, c’est maintenant.
\item Pour plus d’alternatives libres.
\item Parce que les gentils, c’est nous !
\item Pour décider où vont vos impôts (avec défiscalisation).
\item Parce que l’économie du don rend indépendant.
\item Pour changer le monde ensemble.
\end{itemize}
}
\end{block}
\end{frame}


%------------------------------------------------
\begin{frame}
\frametitle{Framasoft est transparent sur l'usage des dons}

\begin{center}
\includegraphics[scale=0.3]{./images/graphique.png}
\\~\\
\url{https://soutenir.framasoft.org/}
\end{center}
\end{frame}

%----------------------------------------------------------------------------------------
\begin{frame}
\Huge{\centerline{Conclusion}}
\end{frame}

%------------------------------------------------
\begin{frame}
\frametitle{Que retenir?}

\begin{block}{Ce qu'il faut retenir}
\justifying{
\begin{itemize}
\item Utiliser les services de Framasoft
\item Donner à Framasoft
\item Se lancer à son tour (autohebergement...)
\end{itemize}
}
\end{block}
\end{frame}



%----------------------------------------------------------------------------------------
\begin{frame}
\Huge{\centerline{Merci de votre attention.}}
\Huge{\centerline{Place aux questions. Débattons...}}
\end{frame}

\end{document}
